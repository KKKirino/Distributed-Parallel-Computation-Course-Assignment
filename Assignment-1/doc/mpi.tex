\documentclass{article}
  \usepackage{pgfplots}
  \pgfplotsset{compat=1.5.1} % 用来保证双 y 轴时两个 y 轴的 label 正确显示

  \input{../../.template/structure.tex} % Include the file specifying the document structure and custom commands

  % 如果提示没有字体,请修改此处的字体路径为当前编译文件的相对路径
  \setmainfont[ExternalLocation=../../.template/]{STZHONGS.ttf}
  
  %----------------------------------------------------------------------------------------
  %	ASSIGNMENT INFORMATION
  %----------------------------------------------------------------------------------------
  
  \title{\Large 分布式并行计算课程作业 \#1 \\
  \LARGE 埃拉托斯特尼素数筛选算法并行及性能优化} % Title of the assignment
  
  \author{杨睿妮 \texttt{(2018011205014)} \\ \url{yangruinii@foxmail.com}} % Author name and email address
  
  \date{电子科技大学 --- \today} % University, school and/or department name(s) and a date
  
  %----------------------------------------------------------------------------------------
  
  \begin{document}
  
  \maketitle % Print the title

  \section{实验说明}
  \begin{itemize}
    \item 使用 MPI 编程实现埃拉托斯特尼筛法并行算法。
    \item 对程序进行性能分析以及调优
  \end{itemize}

  \section{实验环境}
  实验环境如表 \ref{tab:env} 所示。
  \begin{table}[htbp]
    \centering
    \caption{实验环境配置表}
    \label{tab:env}
    \begin{tabular}{ll}
      \hline
      处理器 & lntel(R) Core(TM) i5-8300H CPU @2.30GHz \\
      \hline
      内存 & 16GB \\
      \hline
      MPI 库 & Open MPI 2.1.1\\
      \hline
      操作系统 & Ubuntu 18.04 LTS \\
      \hline
    \end{tabular}
  \end{table}
  
  其中 Open MPI 环境使用如下命令配置:
  \begin{commandline}
    \begin{verbatim}
$ sudo apt install openmpi-bin openmpi-common openmpi-doc libopenmpi-dev
    \end{verbatim}
\end{commandline}

并使用以下命令编译和运行基准代码,以使用四线程求解 1000 以内的素数个数为例:
\begin{commandline}
  \begin{verbatim}
$ mpic++ baseline.c -o baseline.o
$ mpirun -np 4 ./baseline.o 1000
  \end{verbatim}
\end{commandline}

  \section{程序说明}
  \subsection{基准代码 Bug 修改}
  首先我们编译基准代码,以试运行求解 $10^9$ 以内的素数个数。发现在单线程下,可以正常求解 $10^9$ 以内素数个数,但是在多线程的情况下,只能够求解 $10^8$ 以内的素数个数,在求解 $10^9$及以上的规模时,程序输出 \lstinline{Cannot allocate enough memory},查看基准代码后,发现在分配 \verb|marked| 数组时出现内存申请错误。

  进一步定位错误,发现 \verb|size| 变量存在溢出问题。解决方法为修改与 \verb|size| 相关的变量为 \verb|long| 类型。 修改后的相关源码如下:
  \begin{file}[baseline.cpp]
    \begin{lstlisting}[language=C++]
// ...
long high_value; 
long low_value;
long size;
long proc0_size;
// ...
proc0_size = (n - 1) / p;
low_value = 2 + id * proc0_size;
high_value = 1 + (id + 1) * proc0_size;
size = high_value - low_value + 1;
// ...
    \end{lstlisting}
  \end{file}

  至此基准代码可以正常求解 $10^9$ 级别素数的个数问题。

  \subsection{基准代码性能分析}
  \begin{figure}[h]
    \centering
    \begin{tikzpicture}
      \pgfplotsset{
        width=0.8\textwidth,
        height=0.25\textwidth,
        set layers,
      }
      \begin{axis}[
        scale only axis,
        xmin = 0, xmax = 17,
        xlabel = 进程数量,
        ylabel = 耗时(s),
        axis y line*= left,
      ]
        \addplot+[smooth]
         coordinates
        {
          (1,17.1) (2,9.2) (4,7.2)
          (8,6.1) (16,5.1)
        };
        \label{plot:base_time}
        % \addlegendentry{耗时}
      \end{axis}

      \begin{axis}[
        scale only axis,
        ymin = 0.5, ymax = 5,
        xmin = 0, xmax = 17,
        axis y line* = right,
        ylabel = 加速比
      ]
        \addlegendimage{/pgfplots/refstyle=plot:base_time}\addlegendentry{耗时}
        \addplot+[
          smooth,
          color=red
        ]
        coordinates
        {
          (1,1) (2,1.8) (4,2.1)
          (8,2.5) (16,3.4)
        };
        \addlegendentry{加速比}
      \end{axis}
    \end{tikzpicture}
    \label{fig:baseline}
    \caption{基准代码在不同线程下的运行时间和加速比}
  
  \end{figure}

  \subsection{性能优化:去除偶数优化}

  \subsection{性能优化:消除广播优化}

  \subsection{性能优化:Cache 优化}
\end{document}