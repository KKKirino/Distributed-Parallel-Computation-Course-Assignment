\documentclass[11pt]{article}
  \usepackage{pgfplots}
  \pgfplotsset{compat=1.5.1} % 用来保证双 y 轴时两个 y 轴的 label 正确显示

  \input{../../.template/structure.tex} % Include the file specifying the document structure and custom commands

  % 如果提示没有字体,请修改此处的字体路径为当前编译文件的相对路径
  \setmainfont[ExternalLocation=../../.template/]{STZHONGS.ttf}

  % 用于在代码内显示中文
  \usepackage{xeCJK}
  \setCJKmonofont[ExternalLocation=../../.template/]{STZHONGS.ttf}

  \usepackage[font=small,labelfont=bf,tableposition=top]{caption}

  \DeclareCaptionLabelFormat{andtable}{#1~#2  \&  \tablename~\thetable}
  
  %----------------------------------------------------------------------------------------
  %	ASSIGNMENT INFORMATION
  %----------------------------------------------------------------------------------------
  
  \title{\Large 分布式并行计算课程作业 \#1 \\
  \LARGE 埃拉托斯特尼素数筛选算法并行及性能优化} % Title of the assignment
  
  \author{杨睿妮 \texttt{(2018011205014)} \\ \url{yangruinii@foxmail.com}} % Author name and email address
  
  \date{电子科技大学 --- \today} % University, school and/or department name(s) and a date
  
  %----------------------------------------------------------------------------------------
  
  \begin{document}
  
  \maketitle % Print the title

  \section{实验说明}
  \begin{itemize}
    \item 使用 MPI 编程实现埃拉托斯特尼筛法并行算法。
    \item 对程序进行性能分析以及调优
  \end{itemize}

  \section{实验环境}
  实验环境如表 \ref{tab:env} 所示。
  \begin{table}[h]
    \centering
    \caption{实验环境配置表}
    \label{tab:env}
    \begin{tabular}{ll}
      \hline
      处理器 & lntel(R) Core(TM) i5-8300H CPU @2.30GHz \\
      \hline
      内存 & 16GB \\
      \hline
      MPI 库 & Open MPI 2.1.1\\
      \hline
      操作系统 & Ubuntu 18.04 LTS \\
      \hline
    \end{tabular}
  \end{table}
  
  其中 Open MPI 环境使用如下命令配置:
  \begin{commandline}
    \begin{verbatim}
$ sudo apt install openmpi-bin openmpi-common openmpi-doc libopenmpi-dev
    \end{verbatim}
\end{commandline}

并使用以下命令编译和运行基准代码,以使用四线程求解 1000 以内的素数个数为例:
\begin{commandline}
  \begin{verbatim}
$ echo 0 | sudo tee /proc/sys/kernel/yama/ptrace_scope # 关闭报错显示
$ mpic++ baseline.c -o baseline.o
$ mpirun -np 4 ./baseline.o 1000
  \end{verbatim}
\end{commandline}

  \section{程序说明}
  \subsection{基准代码 Bug 修改}
  首先我们编译基准代码,以试运行求解 $10^9$ 以内的素数个数。发现在单线程下,可以正常求解 $10^9$ 以内素数个数,但是在多线程的情况下,只能够求解 $10^8$ 以内的素数个数,在求解 $10^9$及以上的规模时,程序输出 \lstinline{Cannot allocate enough memory},查看基准代码后,发现在分配 \verb|marked| 数组时出现内存申请错误。

  进一步定位错误,发现 \verb|size| 变量存在溢出问题。解决方法为修改与 \verb|size| 相关的变量为 \verb|long| 类型。 修改后的相关源码如下:
  \begin{file}[baseline.cpp]
    \begin{lstlisting}[language=C++]
// ...
long count = 0;
long global_count = 0; // 正确初始化两个 count 变量
long high_value; 
long low_value;
long size;
long proc0_size;
// ...
proc0_size = (n - 1) / p;
low_value = 2 + id * proc0_size;
high_value = 1 + (id + 1) * proc0_size;
size = high_value - low_value + 1;
// ...
    \end{lstlisting}
  \end{file}

  至此基准代码可以正常求解 $10^9$ 级别素数的个数问题。

  \subsection{重构基准代码}
  由于实验指导书中给出的基准代码在确定起始素数时较为繁琐(即附录\ref{apd:fixed} 的 80 - 88 行代码),为了方便进行后续优化,我们对基准代码进行了重构,重构后的完整代码见附录 \ref{apd:refactored}。重构后的程序流程图如下:

  \begin{figure}[h]
    \centering
    \begin{tikzpicture}[node distance=10pt]
      \node[draw, rounded corners]                        (input)   {输入 n};
      \node[draw, below=of input]                         (step 1)  {初始化 MPI 环境并声明相关变量};
      \node[draw, below=of step 1]                        (step 2)  {计算当前线程的区间 \verb|(low_range, high_range)|};
      \node[draw, below=of step 2]                   (step 3)  {寻找下一个素数 \verb|prime|,并广播};
      \node[draw, diamond, aspect=2, below=of step 3]     (choice)  {\verb|prime <= n|?};
      \node[draw, right=30pt of choice]                   (step 3_1)  {使用 \verb|prime| 在区间内进行标记};
      \node[draw, below=of choice]  (step 4)     {统计素数个数并汇总};
      \node[draw, rounded corners, below=of step 4]  (end)     {结束};
      
      \graph{
        (input) -> (step 1) -> (step 2) -> (step 3) -> (choice) ->["Yes"left] (step 4) -> (end);
        (choice) ->["No"] (step 3_1) ->[to path={|- (\tikztotarget)}] (step 3);
      };
    \end{tikzpicture}
  \end{figure}
  后续的偶数优化、消除广播等改动均在重构的代码基础上进行修改。

  \subsection{基准代码性能分析}
  \begin{figure}[h]
    \centering
    \begin{tikzpicture}
      \pgfplotsset{
        width=0.8\textwidth,
        height=0.2\textwidth,
        set layers,
      }
      \begin{axis}[
        scale only axis,
        xmin = 0, xmax = 17,
        xlabel = 进程数量,
        ylabel = 耗时(s),
        axis y line*= left,
      ]
        \addplot+[smooth]
         coordinates
        {
          (1, 15.76) (2, 10.15) (4, 8.76) (8, 8.07) (16, 8.33)
        };
        \label{plot:base_time}
        % \addlegendentry{耗时}
      \end{axis}

      \begin{axis}[
        scale only axis,
        ymin = 0.5, ymax = 5,
        xmin = 0, xmax = 17,
        axis y line* = right,
        ylabel = 加速比
      ]
        \addlegendimage{/pgfplots/refstyle=plot:base_time}\addlegendentry{耗时}
        \addplot+[
          smooth,
          color=red
        ]
        coordinates
        {
          (1, 1.00) (2, 1.55) (4, 1.80) (8, 1.95) (16, 1.89)
        };
        \addlegendentry{加速比}
      \end{axis}
    \end{tikzpicture}
    \caption{基准代码在不同线程下的运行时间和加速比}
    \label{fig:baseline}
  \end{figure}

  \begin{table}[h]
    \centering
    \caption{基准代码性能数据}
    \label{tab:baseline}
    \begin{tabular}{cccccc}
      \hline
      线程数 & 1 & 2 & 4 & 8 & 16 \\
      \hline
      耗时(s) & 15.76 & 10.15 & 8.76 & 8.07 & 8.33 \\
      \hline
      加速比 & 1.00 & 1.55 & 1.80 & 1.95 & 1.89 \\
      \hline
    \end{tabular}
  \end{table}

  我们分别在 $1, 2, 4, 8, 16$ 进程规模下运行了基准代码,并得到了程序运行时间和加速比,结果如图 \ref{fig:baseline} 所示,可以看到,当线程数量较少时,多线程可以有效减少运行时间,但是随着线程数增多,加速比并没有随着线程数量线性增长,这是由于基准代码的广播和线程同步机制,每完成一轮素数筛选,其余线程均需要等待线程 0 广播下一个素数,从而造成额外的时间开销。

  \pagebreak
  \section{性能优化}
  \subsection{性能优化:去除偶数}
  我们使用对基线代码重构并在不同的地方对偶数进行过滤,主要有以下四处修改:
  \begin{enumerate}[itemindent=1em]
    \item 在确定作为筛的素数时,直接从 3 开始,从而跳过偶数;
    \item 在对素数的倍数进行标记时,直接跳过偶数;
    \item 在寻找下一个素数筛时,仅对奇数位标志进行检验;
    \item 在统计素数个数时,同样地只对奇数进行计算。
  \end{enumerate}

  核心代码如下所示,完整代码可查阅附录 \ref{apd:no_even}。

  \begin{file}[no\_even.cpp]
    \begin{lstlisting}[language=C++]
// ...
bool *marked = new bool[proc_array_size](); // 标记数组

// 1. 直接从 3 开始进行计算,从而跳过偶数判定
for (long prime = 3; prime * prime <= n;)
{
  // 确定本线程的开始素数
  long first = (low_range / prime) * prime;
  for (int x = first; x < high_range; x += prime)
  {
    // 跳过没有在范围内的数字,以及素数本身
    if (x < low_range || x == prime || x % 2 == 0)
      continue; // 2. 跳过偶数,40% 左右的性能提升
    marked[x - low_range] = 1;
  }

  if (!pid)
  {
    do {
      prime += 1 + (prime % 2); // 3. 寻找素数时跳过偶数
    } while (prime < high_range && marked[prime]);
  }

  // 广播找到的下一个素数
  MPI_Bcast(&prime, 1, MPI_INT, 0, MPI_COMM_WORLD);
}

// 保证从 3 开始计数
for (int i = low_range; i < high_range; i += 1 + (i % 2))
{
  if (i < 3) continue;
  // 4. 统计数字时同样跳过偶数
  count += !marked[i - low_range] && (i % 2);
}
// ...
    \end{lstlisting}
  \end{file}

  \begin{figure}[h]
    \centering
    \begin{tikzpicture}
      \pgfplotsset{
        width=0.8\textwidth,
        height=0.2\textwidth,
        set layers,
      }
      \begin{axis}[
        scale only axis,
        xmin = 0, xmax = 17,
        xlabel = 进程数量,
        ylabel = 耗时(s),
        axis y line*= left,
      ]
        \addplot+[smooth]
         coordinates
        {
          (1, 10.33) (2, 6.56) (4, 5.26) (8, 5.00) (16, 5.14)
        };
        \label{plot:base_time}
        % \addlegendentry{耗时}
      \end{axis}

      \begin{axis}[
        scale only axis,
        ymin = 0.5, ymax = 5,
        xmin = 0, xmax = 17,
        axis y line* = right,
        ylabel = 加速比
      ]
        \addlegendimage{/pgfplots/refstyle=plot:base_time}\addlegendentry{耗时}
        \addplot+[
          smooth,
          color=red
        ]
        coordinates
        {
          (1, 1.00) (2, 1.58) (4, 1.96) (8, 2.07) (16, 2.01)
        };
        \addlegendentry{加速比}
      \end{axis}
    \end{tikzpicture}
    \caption{去除偶数后在不同线程下的运行时间和加速比}
    \label{fig:no_even}
  \end{figure}

  \begin{table}[h]
    \centering
    \caption{去除偶数后的性能数据}
    \label{tab:no_even}
    \begin{tabular}{cccccc}
      \hline
      线程数 & 1 & 2 & 4 & 8 & 16 \\
      \hline
      耗时(s) & 10.33 & 6.56 & 5.26 & 5.00 & 5.14 \\
      \hline
      加速比 & 1.00 & 1.58 & 1.96 & 2.07 & 2.01 \\
      \hline
    \end{tabular}
  \end{table}
  图 \ref{fig:no_even} 和表 \ref{tab:no_even} 展示了去除偶数后的运行时间和加速比。相对于基准代码,跳过偶数可以带来 30\% - 40\% 的性能提升。


  \subsection{性能优化:消除广播}
  在偶数优化后的代码基础上,我们在每个线程内预先地计算前 $\sqrt{n}$ 个素数,然后再对每个区间端进行筛选,从而免去了素数广播耗时,预计算部分为:
  \begin{file}[no\_cast.cpp]
    \begin{lstlisting}[language=C++]
// ...
// 计算前 sqrt(n) 个素数
int sqrtn = (int)sqrtl(n) + 1;
bool *sqn_mark = new bool[sqrtn]();
for (int prime = 2; prime * prime < sqrtn; prime += 1) {
  if (marked[prime]) continue;
  for (int i = 2; i * prime < sqrtn; i += 1) {
    sqn_mark[i * prime] = 1;
  }
}
// ...
    \end{lstlisting}
  \end{file}
  然后在寻找下一个素数时,只需要在 \verb|sqn_mark| 数组中寻找即可:
  \begin{file}[no\_cast.cpp]
    \begin{lstlisting}[language=C++]
// 找到下一个素数
do {
  prime += 1 + (prime % 2); // 寻找素数时跳过偶数
} while (prime < high_range && sqn_mark[prime]);
    \end{lstlisting}
  \end{file}
  \begin{figure}[h]
    \centering
    \begin{tikzpicture}
      \pgfplotsset{
        width=0.8\textwidth,
        height=0.2\textwidth,
        set layers,
      }
      \begin{axis}[
        scale only axis,
        xmin = 0, xmax = 17,
        xlabel = 进程数量,
        ylabel = 耗时(s),
        axis y line*= left,
      ]
        \addplot+[smooth]
         coordinates
        {
          (1, 11.22) (2, 6.82) (4, 5.75) (8, 4.88) (16, 4.59)
        };
        \label{plot:base_time}
        % \addlegendentry{耗时}
      \end{axis}

      \begin{axis}[
        scale only axis,
        ymin = 0.5, ymax = 5,
        xmin = 0, xmax = 17,
        axis y line* = right,
        ylabel = 加速比
      ]
        \addlegendimage{/pgfplots/refstyle=plot:base_time}\addlegendentry{耗时}
        \addplot+[
          smooth,
          color=red
        ]
        coordinates
        {
          (1, 1.00) (2, 1.64) (4, 1.95) (8, 2.30) (16, 2.44)
        };
        \addlegendentry{加速比}
      \end{axis}
    \end{tikzpicture}
    \caption{消除广播后在不同线程下的运行时间和加速比}
    \label{fig:no_cast}
  \end{figure}

  \begin{table}[h]
    \centering
    \caption{消除广播后的性能数据}
    \label{tab:no_cast}
    \begin{tabular}{cccccc}
      \hline
      线程数 & 1 & 2 & 4 & 8 & 16 \\
      \hline
      耗时(s) & 11.22 & 6.82 & 5.75 & 4.88 & 4.59 \\
      \hline
      加速比 & 1.00 & 1.64 & 1.95 & 2.30 & 2.44 \\
      \hline
    \end{tabular}
  \end{table}

  由于线性筛法的时间复杂度为 $O(n \log \log n)$ \footnote{证明见 \url{https://oi-wiki.org/math/sieve/#_6}。},所以预计算部分引入的额外耗时为 $O(\sqrt{n} \log \log n)$ 级别,并不会对整体复杂度引起数量级的变化。

  图 \ref{fig:no_cast} 和表 \ref{tab:no_cast} 展示了消除广播后的运行时间和加速比。当线程数较少时,相对于优化前的代码,其运行耗时反而更久,其原因是预计算前 $\sqrt{N}$ 区间内的素数会引入额外耗时,而当线程数量增加后,避免广播同步带来的耗时减少抵消了预计算的耗时,总耗时得以下降,最终可以带来 10\% 的性能提升。

  \subsection{性能优化:Cache 优化}
  在偶数优化章节中,我们发现跳过偶数可以带来接近 40\% 的性能提升,进一步探究可以发现,整个算法最耗时的部分在于对 \verb|marked| 数组的访问,经过埋点计时发现,对 \verb|marked| 数组的访问耗时占了整个算法总耗时的 90\% 左右,所以通过优化 \verb|marked| 的访存,我们可以进一步提升程序性能。

  我们编写了一段简单的程序来观察申请内存区域的大小与申请和访问耗时之间的关系,如下所示:
  \begin{file}[test\_mem.cpp]
    \begin{lstlisting}[language=C++]
// 指定总循环次数和内存空间大小
long n = atol(argv[1]);
long seg_size = atol(argv[2]);

long* arr = new long[seg_size]();

for (long j = 0; j < n; j += seg_size) {
  for (long i = 0; i < seg_size; i += 1) {
    if (arr[i]);
  }
}
  \end{lstlisting}
  \end{file}

  通过固定总访问次数,然后遍历空间大小,我们得到访问耗时和内存大小之间的数据趋势,图 \ref{tab:mem} 展示了详细的测试结果,我们发现随着 \verb|arr| 数组的增大,上述代码运行总耗时也在不断增加,而当申请的数组大小在 $10^8 - 10^9$ 级别时,耗时大幅增加。

  \begin{table}[h]
    \centering
    \caption{不同大小的内存访问耗时}
    \label{tab:mem}
    \begin{tabular}{ccccccc}
      \hline
      数组大小 & $10^4$ & $10^5$ & $10^6$ & $10^7$ & $10^8$ & $10^9$ \\
      \hline
      总耗时(s) & 1.812 & 1.995 & 2.356 & 2.518 & 3.111 & 9.218 \\
      \hline
    \end{tabular}
  \end{table}

  由于本实验的算法运行在 $10^9$ 规模的数据上,尽管在 16 线程下,其申请的 \verb|marked| 数组大小也在 $10^8$ 级别,所以减小 \verb|marked| 数组大小可以有效降低运行耗时。

  经过参数调优,我们发现当 \verb|marked| 数组的大小设置为 $10^6$ 时,可以在实验机器上得到最优耗时,进一步分析可知,实验机器的 L3 缓存大小为 8MB,当线程数开到 8 时,每个线程的可用的最大缓存为 1MB,所以缓存大小设置为 $10^6$ 以内为最佳。

  附录 \ref{apd:better_cache} 展示了进行 Cache 优化后的代码,其中核心代码为:

  \begin{file}[better\_cache.cpp]
    \begin{lstlisting}[language=C++]
// 使用分块算法优化 cache 访存
long seg_size = 1000000;
bool *marked = new bool[seg_size](); // 标记数组
for (int i = low_range; i < high_range; i += seg_size) {
  count += solve(i, min(high_range, i + seg_size), marked,
                  seg_size, sqn_mark, sqrtn);
}
  \end{lstlisting}
  \end{file}

  其中 \verb|solve| 函数根据预计算的素数数组来对指定区间的数字进行过滤,并返回该区间的素数个数,我们声明好固定大小的 \verb|marked| 数组,并在 \verb|solve| 函数中反复使用该数组,从而进一步降低算法运行耗时。

  \begin{figure}[h]
    \centering
    \begin{tikzpicture}
      \pgfplotsset{
        width=0.8\textwidth,
        height=0.2\textwidth,
        set layers,
      }
      \begin{axis}[
        scale only axis,
        xmin = 0, xmax = 17,
        xlabel = 进程数量,
        ylabel = 耗时(s),
        axis y line*= left,
      ]
        \addplot+[smooth]
         coordinates
        {
          (1, 9.79) (2, 4.98) (4, 2.66) (8, 2.09) (16, 2.27)
        };
        \label{plot:base_time}
        % \addlegendentry{耗时}
      \end{axis}

      \begin{axis}[
        scale only axis,
        ymin = 0.5, ymax = 5,
        xmin = 0, xmax = 17,
        axis y line* = right,
        ylabel = 加速比
      ]
        \addlegendimage{/pgfplots/refstyle=plot:base_time}\addlegendentry{耗时}
        \addplot+[
          smooth,
          color=red
        ]
        coordinates
        {
          (1, 1.00) (2, 1.97) (4, 3.68) (8, 4.69) (16, 4.32)
        };
        \addlegendentry{加速比}
      \end{axis}
    \end{tikzpicture}
    \caption{Cache 优化后在不同线程下的运行时间和加速比}
    \label{fig:cache}
  \end{figure}
  \begin{table}[h]
    \centering
    \caption{Cache 优化后的性能数据}
    \label{tab:cache}
    \begin{tabular}[b]{cccccc}
      \hline
      线程数 & 1 & 2 & 4 & 8 & 16 \\
      \hline
      耗时(s) & 9.79 & 4.98 & 2.66 & 2.09 & 2.27 \\
      \hline
      加速比 & 1.00 & 1.97 & 3.68 & 4.69 & 4.32 \\
      \hline
    \end{tabular}
  \end{table}

  图 \ref{fig:cache} 和表 \ref{tab:cache} 展示了进行 Cache 优化后的结果,相对于优化之前,Cache 优化带来了超过 100\% 的性能提升,并且当线程数量小于 8 时,加速比可以近似地线性增加。当线程超过 8 时,我们发现程序运行效率反而会下降,这是因为实验机器的逻辑处理器为 8 个核心,超过 8 线程时,线程切换会引入额外的运行开销,从而增加耗时。

  \subsection{最终调优}
  最后我们进一步重构了代码,将内层循环的特判以及循环索引计算等提到循环外面,来减少计算量,并且将预计算的素数数组进一步紧凑表示,从而无需通过 \verb|do...while| 循环的方式去寻找下一个素数,从而节省部分运算量,最终代码如附录 \ref{apd:final_version} 所示。

  \pagebreak
  \section{实验总结}
  本实验以实现筛法求解素数为目标,不断地优化程序性能以达到最佳性能。实验指导书提供的初始版本代码存在数值溢出 BUG,修复之后可以正常求解,但运行速度十分缓慢。为了方便进行进一步优化,我们重构了原始代码,并且开始进行优化。

  偶数优化可以将整体的运算量直接减半,从而使得实验性能有巨大提升,我们在执行每个循环时,都对循环步长进行了特殊处理,保证循环从奇数索引开始遍历,并设置步长为 2,从而使得程序性能达到最佳。

  消除广播可以使得线程之间的通信减少,不必等待线程 0 广播下一个素数,其主要思想是在每个线程内预计算前 $\sqrt{N}$ 区间内的素数,我们可以从复杂度角度证明预计算不会带来数量级的额外耗时,但是当线程数较少时,预计算带来的性能提升并不明显。

  为了解决高速 CPU 和低速存储器之间的速度不匹配问题,现代 CPU 均配备了高速缓存,适当利用这些缓存可以进一步提升算法性能。我们在拥有 8MB L3 缓存的机器上,使用 1MB 以内的内存大小可以最大程度地利用这些高速缓存,并达到最优性能。

  \begin{table}[h]
    \centering
    \caption{各种优化后的耗时(s)}
    \label{tab:cache}
    \begin{tabular}[b]{cccccc}
      \hline
      优化名称 & 1 & 2 & 4 & 8 & 16 \\
      \hline
      Baseline & 14.98 & 10.02 & 8.70 & 8.21 & 7.95\\
      No Even & 9.99 & 6.56 & 5.51 & 4.99 & 4.66\\
      No Cast & 10.84 & 7.06 & 5.34 & 4.71 & 4.40\\
      Better Cache & 5.80 & 3.04 & 1.85 & 1.13 & 1.03\\
      Final Version & 3.79 & 1.98 & 1.59 & 0.91 & 0.81\\
      \hline
    \end{tabular}
  \end{table}

  \section{实验心得}
  通过不断地对 MPI 程序进行优化,我们对多线程加速技术有了更深层次的理解,减少计算量、减少线程通信、高效利用计算机缓存等手段在实际的工程项目中具有很强的借鉴意义,同时 MPI 多线程加速技术也是一种非常高效的性能加速手段。

  \pagebreak
  \appendix
  % 附录代码开启行号
  \lstset{
	  numbers=left,
  }

  \section{附录:修复 Bug 后的基准代码}
  \label{apd:fixed}
  \lstinputlisting[language=C++]{../code/example.cpp}
  \section{附录:重构后的基准代码}
  \label{apd:refactored}
  \lstinputlisting[language=C++]{../code/baseline.cpp}
  \section{附录:偶数优化}
  \label{apd:no_even}
  \lstinputlisting[language=C++]{../code/no_even.cpp}
  \section{附录:消除广播}
  \label{apd:no_cast}
  \lstinputlisting[language=C++]{../code/no_cast.cpp}
  \section{附录:优化 Cache}
  \label{apd:better_cache}
  \lstinputlisting[language=C++]{../code/better_cache.cpp}
  \section{附录:最终代码}
  \label{apd:final_version}
  \lstinputlisting[language=C++]{../code/final_version.cpp}
\end{document}